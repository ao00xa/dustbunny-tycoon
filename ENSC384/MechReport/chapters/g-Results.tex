\section{Results}

The final design was built and tested. 
It was found to meet all criteria.
A comparison to predicted results for both horizontal and vertical loading is presented in the two sections below.

\subsection{Vertical Deflection and Performance}

During testing, the vertical deflection under a load of 5 N was observed to be 3.10 mm.
Results obtained through a finite element analysis performed in SolidWorks were comparable with a predicted deflection of 3.95 mm.
The difference between the tested result and that generated by SolidWorks is not unexpected and likely caused by imperfect modeling of end conditions, material properties and joint properties.
%	Something more about joints?
Analytical and finite element analysis performed in MATLAB predicted deflections of orders of magnitude less than those observed.
This discrepancy is likely due to the faulty assumptions about end-conditions of the truss.
%	Has this been discussed? Where? TODO: reference section
% More (?)

\subsection{Horizontal Deflection and Performance}

During testing, the horizontal deflection under a load of 5 N was observed to be 0.21 mm.
The SolidWorks finite element analysis predicted a much lower deflection of 0.042 mm.
The SolidWorks result is similar to that predicted by the MATLAB finite element analysis which suggested a deflection of 0.0409 mm.
Additionally, the analytical result from section \ref{sec:IDontCare} gives a virtually identical result of 0.0409 mm. 
There are many factors the analysis techniques do not account for which could cause the discrepancy between predicted and actual results. %Used discrepancy twice
For example, the motor mount could have shifted or the screws attaching the truss bent; each of which is not considered by the analysis.
	%TERRIBLE sentence.
It is also possible the assumptions made when analyzing the structure are invalid.