\section{Limitations of Simulation}
\label{sec:SimulationLimitations}

The controller presented in this report was designed almost exclusively by experimental and heuristic methods.
Virtually no result produced via analytical means or simulation was incorporated into the controller as they were found to be generally inaccurate.
Although it is likely controller gains, switching thresholds, filter pass-bands, stability conditions and other tunable parameters could be found analytically and/or modeled accurately, such a process was found to be impractical and unnecessary, especially given the project time constraints. 
The following list details various reasons analytical analysis and simulation was impractical; many follow from the fact that the controller operates in the near optimal, nonlinear region of the hardware. 

\begin{itemize}
%\item the simulations are done within the bounds of the physical system constraints. 
%\item simulating the nonlinear limitations is very difficult and it requires a lot of work to achieve accurate results. 
\item Motor acceleration was seen to be limited by current saturation within the power-supply, not the motor. The saturating characteristics of the power-supply are complex, difficult to measure and model, and have a large impact on system performance. 
\item Software PWM is highly dependent on sampling rate and hardware speed characteristics. Mixing sampling rates within a simulation introduces secondary complexities.
\item Position is a discreet variable due to the encoder being used.
\item Mechanical vibration was seen to be a significant issue and dictated certain controller design features. These vibrations are difficult to simulate.
\item External forces with non-linear characteristics, such as torsion in the magnet wire are unknown.
\end{itemize}
