\section{Problem Definition}

\subsection{Determination of Truss Inertia}

The project's truss inertia, $J_{truss}$ (mass moment of inertia about the rotational axis), is a fundamental quantity that is dependent upon the design specifications of the truss and is initially unknown. 
Since  $J_{truss}$ appears in mathematical equations associated with the controller, it must be determined in order to allow for proper position/speed control of the system.
One method for calculating the truss inertia is to model and analyze a given truss design using 3-D CAD software.
A second way, which can also be used to check modeled values, it to create an experimental test, using parameters provided by the motor manufacturer, to estimate the  $J_{truss}$ value. 
A possible inaccuracy with this method is that not all system parameters are necessarily known with reasonable certainty. 
In this report, an experimental method is described and the results are compared with theoretical values determined from the modeling performed at an earlier stage of the project.

\subsection{Controller Design}

%TODO: Fix me.
%Controller has three parameters to choose $K_p$, $K_i$, $K_d$.
%Choice of parameters changes system response.
%Choose based on desired response - in our case overshoot and settling time.

%\emph{Insert simple PID block diagram here}

In order to obtain a suitable overall system response, a feedback controller of some type is needed.
This report describes the design of a proportional, integral, derivative controller.
Such controller is accurate and relatively easier to implement and simulate than other types. 
Furthermore, in most cases the more complex controller is, the more it costs, the less reliable it is, and it is more difficult to design. 
A PID controller requires careful choice of its gains, $K_p$, $K_d$, and $K_i$, to achieve the desired response and to meet design specifications. 

\subsection{Simulation} 

To verify the system response given controller gains determined using system theory, a simulation is invaluable.
Simulation allows additional experimentation based on a more realistic model which incorporates saturation in an effort tune gain parameters. 
A simulation also allows experimentation with no possibility of damaging physical components. %Yuck 

\subsection{Amplifier Control} 

The PWM amplifier being used was found to be generally nonlinear, and difficult to control.
A new approach to amplifier control is presented which eliminates a large portion of the circuitry. 

