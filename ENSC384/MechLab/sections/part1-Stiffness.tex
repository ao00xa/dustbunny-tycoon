\subsubsection{Net Stiffness and Efficiency}
%6) Calculate the net stiffness of each truss structure (applied force vs deflection) and determine its overall efficiency (stiffness/structure weight). To calculate the weight of each structure, assume brass has a density of 8.49 g/cm3. -> 8490 kg / (m^3)
%7)	Observe how the structure efficiency changes with the truss designs. What design features make the truss structure more or less efficient? 

The relative stiffness of each truss is given in table \ref{tbl:stiffness}.
%TODO: more info
It can clearly be seen that the fifth design has the greatest efficiency by a large margin.
The primary feature distinguishing it from the other designs is its height of 10 cm instead of the other designs having heights of 5 cm.

\begin{table*}[hp]
	\centering
	\caption{Stiffness and Efficiency}
	\label{tbl:stiffness}
	\vspace{6pt}
	\begin{tabular}{ccccccc}
		\toprule
		Design &  \multicolumn{3}{c}{Stiffness} & \multicolumn{3}{c}{Efficiency} \\
		\cmidrule(r){2-4}
		\cmidrule(r){5-7}
		 & Applied Force & Deflection & Stiffness & Material Length & Weight  & Efficiency \\
		\midrule
		1 & 1 N & 0.029 mm  & 34.5 & 0.604 m & 40.6 g & 849 \\
		2 & 1 N & 0.0112 mm & 89.3 & 1.374 m & 92.4 g & 967 \\
		3 & 1 N & 0.0107 mm & 93.5 & 1.1242 m & 75.6 g & 1237 \\
		4 & 1 N & 0.0148 mm & 67.6 & 0.8162 m & 54.9 g &  1232 \\
		5 & 1 N & 0.0036 mm & 277.8 & 0.9243 m & 62.1 g & 4471 \\
		%6 & 1 N & 0.0042 mm & 238.1 & 0.9106 m & 62.1 g & 3890 \\
		\bottomrule
	\end{tabular}
\end{table*}

