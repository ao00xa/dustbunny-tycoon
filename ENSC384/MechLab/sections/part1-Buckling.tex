\subsubsection{Buckling}
%Calculate the critical buckling load for each member of the truss structure undergoing compressive load. Note that the length of the member is the distance between the nearest 2 points of attachment with other members (not only the anchor points). Calculate the factor of safety for buckling failure for each truss structure

A factor of safety with regards to buckling is an important design requirement. 
%TODO: reword
Table \ref{tbl:buckling} lists the lengths of every member in each design which experiences compression.
The force for which a member of the specified length will buckle is given.
Eulerian buckling for columns is assumed with a theoretical fixity constant of 1,  corresponding to two pined ends.
The maximum force predicted by the finite element analysis in any member of the given length is listed which may then be used to compute the factor of safety. 
For an additional factor of safety, the smallest Young's modulus for brass is also assumed (96 GPa).


\begin{table*}[hp]
	\centering
	\caption{Force of Buckling - 5 N Load}
	\label{tbl:buckling}
	\vspace{6pt}
	\begin{tabular}{ccccc}
		\toprule
		Design & Member Length & Buckling Force & Maximum Force Predicted & Factor of Safety \\
		\midrule
		1 & 30 cm & 52.5 N & 30.4 N & 1.72 \\
		\midrule
		2 & 5 cm & 1891 N & 25 N & 75.6 \\
		 & 7.07 cm & 946 N & 7.07 N & 133.7 \\
		\midrule
		3 & 7.07 cm & 946 N & 7.07 N & 133 \\
		 & 10 cm & 472 N & 25 N & 18.9 \\
		\midrule
		4 & 15 cm & 210 N & 15 N & 14 \\
		 & 15.8 cm & 189 N & 15.8 N & 12 \\
		\midrule
		 & 10 cm & 472 N & 15 N & 31.5 \\
		5 & 14 cm & 241 N & 7.071 N  & 34.1 \\
		 & 20 cm & 118 N & 5 N & 23.6 \\
		\bottomrule
	\end{tabular}
\end{table*}
