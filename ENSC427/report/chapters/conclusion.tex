\chapter{Conclusion} 

First and foremost, OPNET has been shown to be a suitable tool for analyzing message ferrying.
The node models created to analyze the specialized \lq{}state monitor\rq{} network were tested and validated.
A more complicated examination was performed with a network model involving ten source nodes and varying numbers of ferry and gateway nodes.
Statistics measuring update success and delay were defined, implemented and collected during an OPNET simulation.

%TODO: Transition

\section{Results}

A number of general conclusions can be drawn from the results presented in \ref{sec:resultsChapter}.
Adding gateways and ferries was seen to reduces delay, reduces the memory requirements of ferries to achieve a desired success rate, and decreases variability in delay (see \ref{sec:results_success_s2}).
As such, any message ferrying network should have a maximum number of ferries and gateways.
Success rate was seen to marginally improve when enabling source storage.
This improvement is expected to increase with additional ferries.
As such, it may be concluded that networks with few ferries need not implement this feature, however it should enabled for networks with many ferries.

%NEED SOMETHING MORE

%An general discussion of results.
%Include a discussion on the feasibility of a practical implementation and what adoption threshold would be needed.

\section{Future Work}

There are four main categories for future work and improvements to the OPNET model in order to study task oriented message ferrying.

\subsection{Algorithm Improvements}

Many aspects of the ferrying algorithm implemented in this network are simplistic.
For example, there is no reverse communication from the gateway to message ferries indicating updates have been delivered and update messages may be discarded.
The implementation of an update acknowledgment mechanism could significantly improve performance and memory utilization.
Additionally, a more intelligent algorithm used by ferries to discard packets could also improve network performance.

\subsection{Model Improvements}

Many assumptions and simplifications were made when considering update and packet transfer between nodes.
For example, near data transfer was assumed, little to no packet loss, an a defined communication range of 60 meters.
%Why valid
Implementing an existing point to point protocol for wireless data transfer, such as WiFi or ZigBee would provide more realistic results.

\subsection{Statistic Improvements}

Due to the unique nature of the network, common ways to measure network statistics are not valid.
As such, custom code was required to measure all statistics.
Measurement of only two statistics was implemented, delay and update loss.
Adding additional statistics, such as number of active packets in the network and arrival order, would provide additional insights into the networks behaviour.

\subsection{Applicability and Network Model}

The simulation presented involves roughly even source node placement and random ferry movement.
It is unlikely that a real %change real
network would have these characteristics.
Creating and simulating a real-world application of message ferrying would provide more realistic statistics.
%Yuck

%\subsection{Fountain Codes}
%\label{sec:fountain_codes}

%An assumption of the scenario was that application messages can be transmitted in a single network packet.
%A possible solution overcome this limitation without using an ARQ scheme (poorly suited to a delay tolerant network) is the use of fountain error correcting codes (aka rateless erasure codes). 
%See \url{http://en.wikipedia.org/wiki/Fountain_codes}.