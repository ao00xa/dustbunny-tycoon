\chapter{Project Premise \& Model Design} 

%TODO: need to discuss ferry characteristics - task oriented

Any application running over a message ferrying network must have the following characteristics.

\begin{description}
\item[Delay Tolerance: ]
Since data is transported by a physical device, significant delays of minutes to hours must be expected.
\item[Loss Tolerance: ]
Given that ferries have limited memory, loss of data must be expected.
\item[Loss Tolerance: ]
Given that ferries have limited memory, loss of data must be expected.
\item[Small and Independent Messages: ]
Following from the limited memory capacity of ferries and the high probability of packet loss, a reliable method for segmentation and reassembly messages should not be expected. 
Applications should limit the size of messages such that the can be transmitted in their entirety using one network packet.
(See section \ref{sec:fountain_codes} for future work)
\end{description}

Given these criteria, a message ferrying network is unsuitable for many classic networking applications including web browsing, real-time voice or text communication and file transfer.
As such, a very specialized 'state monitoring' network designed for one way communication and non-critical monitoring of remote sensors is considered.

\section{State Monitoring Network}

The application considered for this project consists of a network containing numerous, uniquely identifiable source nodes. 
Each source node has a limited number of properties, in the form of key/value pairs, specifying the property name and its current value.
Properties may change overtime and each change defines a new state for the source node.
For example, a temperature sensor might support a 'temperature' property, the value of which is the current temperature.
Properties do not have to contain a single value and each may be as large as the payload limit of network packets. %Somewhat of a vague statement

The network and message ferrying algorithm is designed to synchronize a central repository with the current state of every source node.
Only the most recent state (or most recent value) for each property is important, not the history of how that property has changed. 
The significance of this characteristic is described below. %TODO: ref section

\section{Network Elements and Terminology}

The following terminology is used with respect to this project

\begin{description}
\item[Message: ]
\item[Source Node: ]
\item[Message Ferry: ]
\item[Gateway: ]
\end{description}