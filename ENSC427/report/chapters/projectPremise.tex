\chapter{Project Scenario \& Model Design} 

Any application running over a message ferrying network must have the following characteristics
\begin{description}
\item[Delay Tolerance: ]
Since data is transported by a physical device, significant delays of minutes to hours must be expected.
\item[Loss Tolerance: ]
Given that ferries have limited memory, loss of data must be expected.
\item[Loss Tolerance: ]
Given that ferries have limited memory, loss of data must be expected.
\item[Small and Independent Messages: ]
Following from the limited memory capacity of ferries and the high probability of packet loss, a reliable method for segmentation and reassembly messages should not be expected. 
Applications should limit the size of messages such that the can be transmitted in their entirety using one network packet.
(See section \ref{sec:fountain_codes} for future work)
\end{description}

Given these criteria, a message ferrying network is unsuitable for many classic networking applications such as web browsing, real-time voice or text communication and file transfer.
As such, a very specialized network designed for 