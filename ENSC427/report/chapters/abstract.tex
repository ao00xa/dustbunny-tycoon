Machine to machine communication has long been considered a feature of the next technological age. 
For many applications, networking options available today are either too expensive or cumbersome to justify the information they are able to provide (such as dedicated wired Ethernet or cellular data modems), or do not provide uniform connectivity (such as WiFi). 
An alternate solution is to use ad-hoc mesh networking. 
Such networks however, require every node to be connect to another and fail when the network is sparse or becomes partitioned.
Message ferrying is a technique which uses physical mobile devices, known as message ferries, as data transport mechanisms between disconnected network nodes or partitioned subnetworks.
This report describes a message ferrying algorithm and simulation model for task oriented ferries created in OPNET which is applicable to a specialized remote sensor network in which a central repository maintains current sensor state.
Update success rate and delay results are presented for two simulations.

%TODO: More

%\emph{References for interim report:}
%~\cite{adhocmsgferry}
%~\cite{hybrid}
%~\cite{Routing}
%~\cite{wearable}
%~\cite{QoSrouting}
%~\cite{efficientrouting}
%~\cite{implement}
%~\cite{book1}