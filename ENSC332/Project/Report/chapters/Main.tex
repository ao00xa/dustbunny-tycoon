\section{Main Program Loop} 
\label{sec:main}

Core logic for the PONG game is implemented in C.
A main function containing an infinite loop is documented as part of appendix \ref{code:main.c}.
This file also contains support functions and code used to control the ball.
The program starts by performing various initialization actions. These include:

\begin{itemize}
\item Setting the data direction register for port A such that it may be used to power and detect keypad button presses.
\item Setting PB0 as an output. It is connected to the OLED module and used for resetting.
\item Initializing the serial communication hardware used for interfacing to the OLED.
\item Resetting and initializing the OLED module.
\item Initializing and displaying the initial score on the LCD display.
\item Initializing the C structures used for tracking each players paddle.
\end{itemize}

After initialization, the program enters an \q{outer} infinite loop then an \q{inner} loop which breaks whenever a player scores.
Each iteration of this inner loop encapsulates an animation step including checking if players are moving their paddles, visually changing paddle locations if necessary, and updating the ball location.
A speed factor is used to account for the limited baud rate supported by the OLED module; see section \ref{sec:baud} for more information.