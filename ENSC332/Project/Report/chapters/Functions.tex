\section{Functions - Calling Assembly From C}

A major component of this project was combining C and assembly code.
Assembly code may be \q{in-lined} within a C function.
For clarity however, all assembly used in this project was abstracted into dedicated functions, callable from C.
Any function written in assembly but callable from C should have a header file with the C function prototype. 
The function name is simply an assembler label within the .asm source file.
The \emph{XDEF} directive must also be used on the function name (assembly label).

To effectively write C functions in assembly, it is important to know how arguments are passed to a function and how values are returned from a function.
By examining assembler code generated by the C compiler, the following rules were determined to be the calling convention for functions.
It is important to note that the rules listed below are by no means exhaustive, were experimentally determined, and are valid only for simple data types such as \emph{int}s and \emph{char}s.
%	Yuck, try and reword.

\begin{itemize}
\item The \emph{CALL} instruction is used to invoke a function (from the assembly compiled C code) and assembly returns when the \emph{RTC} instruction is used. The CALL instruction stores three bytes on the stack.
\item 8 bit return values are stored in the B register.
\item 16 bit return values are stored in the D register (A + B).
\item The last function argument is stored in the D (16 bit) or B (8 bit) register when calling.
\item All other function arguments are pushed onto the stack, ordered from first to second last. When accessing these values a 3 position offset must be used to account for the stack locations used by the \emph{CALL} function.
\item It is very important that any space allocated on the stack by the assembly subroutine be cleaned up before it returns.
\end{itemize}