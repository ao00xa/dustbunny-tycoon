\section{Activity 2}

The second activity involved writing an assembly program which added five numbers and stored the result in register A.
The \emph{ADDA} instruction was used to sum a set of predefined values in-place in accumulator A.
The program terminated by entering an infinite loop.
Stepping through the program line by line using the debugger, the final value of accumulator A was verified.

\subsection{Code}

%\begin{verbatim}
{\footnotesize
\begin{lstlisting}
; Example Assembly Language Program
	ABSENTRY Entry ; absolute assembly application entry point
; Include derivative-specific definitions
	INCLUDE 'mc9s12dg256.inc'
ROMStart EQU $4000 ; absolute address to place my code/constant data
; $4000 is where code ROM starts for 9s12dx256 up to $7fff
; variable/data section
	ORG RAMStart ; RAMStart is defined in mc9s12dj256.inc as $1000
; Insert here your data definition.
Counter DS.W 1 ; set aside 1 word for counter in RAM
FiboRes DS.W 1
; code section
	ORG ROMStart
Entry:
	LDAA #$42 ; First number to be summed
	ADDA #$D ; Second through fifth numbers
	ADDA #$F
	ADDA #$D
	ADDA #$F 
HERE JMP HERE
	END ; if system has monitor program END can be removed
;*********************************************
;* Where to go when reset key is pressed *
;*********************************************
	ORG $FFFE
	DC.W Entry ; Reset Vector
\end{lstlisting}
}