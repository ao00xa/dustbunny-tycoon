\cleardoublepage

\section{Questions}

\subsection{Registers}

Information about each register is given in table \ref{tbl:registers} below.

\begin{table*}[hp]
	\centering
	\caption{Register}
	\label{tbl:registers}
	\vspace{6pt}
	\begin{tabular}{lcccccc}
		\toprule
		 & PC & A & B & X & Y & PORTB \\
		\midrule
		Register size (bits) & 16 & 8 & 8 & 16 & 16 & 8 \\
		Largest Value (Decimal) & 65535 & 255 & 255 & 65535 & 65535 & 255 \\
		Largest Value (Hex) & FFFF & FF & FF & FFFF & FFFF & FF \\
		\bottomrule
	\end{tabular}
\end{table*}

\subsection{Files and Compilation}

The process of translating the assembly program into the binary image which is loaded onto the processor involves the creation of a number of different files.
These files, and the program responsible for their generation, are listed below.

\begin{description}

\item[.asm] - 
This file is the assembly program, written to perform a specific task.
%TODO: More
It is created by the programmer using the CodeWarrior IDE.

\item[.lst] -
The .lst file is generated by the assembler and lists any errors which were detected in the .asm file.
%	+ line numbers

\item[.obj] - 
The .obj file is also generated by the assembler. 
It contains the machine code.

\item[.s19] -
This file is generated by the linker from a given set of object files.
It is a hex file containing the data which is to be downloaded to the processor.

\end{description}