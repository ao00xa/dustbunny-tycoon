\chapter{Open Loop Analysis} 

\section{Stability}

System stability depends on the eigenvalues of matrix \textbf{A} from equation \ref{eq:matA}.
The eigenvalues are shown in equation \ref{eq:evalsAraw}.
For simplicity, initial position is assumed to be $z_0=1$.
	%TODO: Does not change
This yields equation \ref{eq:evalsA}.


\begin{align}
	\label{eq:evalsAraw}
	\lambda_1 &= {\frac {65.4}{\sqrt {- 109\,{\it z_0}- 675.8}}} & \lambda_2 &= -{\frac {65.4}{\sqrt {- 109\,{\it z_0}- 675.8}}}
\end{align}

\begin{align}
	\label{eq:evalsA}
	\lambda_1 &= 0 + 2.33\,j & \lambda_2 &=  0 - 2.33\,j
\end{align}

Since the real part of these values is zero, the system is stable in the sense of Laypunov.
This result is expected as friction was assumed to be zero.

\section{Controllability}


The controllability matrix for this system is shown in equation \ref{eq:control}.
It can easily be seen that the matrix is full rank.
As such, the system is controllable.

\begin{equation}
	\label{eq:control}
	{\bf \Phi_c} =
	 \left[
		\begin{array}{ccc}
			 {\bf B} & | & {\bf AB}\\
		\end{array}
	\right]
	= 
	\left[
		\begin{array}{cc}
			 0 & {\frac {0.138}{m}} \\
			 {\frac {0.138}{m}} & 0 \\
		\end{array}
	\right]
\end{equation}

\section{Observability}

The observability matrix for this system is shown in equation \ref{eq:observe}.
It can easily be seen that the matrix is full rank.
As such, the system is observable.

\begin{equation}
	\label{eq:observe}
	{\bf \Phi_o} = 
	 \left[
		\begin{array}{c}
			 {\bf C} \\
			 {\bf CA} \\
		\end{array}
	\right]
	=
	\left[
		\begin{array}{cc}
			1 & 0 \\
			0 & 1 \\
		\end{array}
	\right]
\end{equation}