\chapter{Linearized Model} 

%A linear system model is desired of the form presented in equation \ref{eq:linform}.

%TODO

%\begin{equation}
%	\label{eq:eqptx}
%\end{equation}

\section{Equilibrium Point}

The system is linearized about an equilibrium point.
This point is given by equation\ \ref{eq:eqptx} where $z_0$ represents the initial position of the magnet. %TODO: x only
%TODO: why
The input equilibrium may be found using equation \ref{eq:syseq2} as shown in \ref{eq:ustar}.

\begin{equation}
	\label{eq:eqptx}
	{\bf x^\star} = 
	\left[
		\begin{array}{c}
			z_{0} \\
			0 \\
		\end{array}
	\right]
\end{equation}

\vspace{6pt}

\begin{equation}
	{\frac {u^\star}{0.000105\,\left(z_0 + 6.2 \right) ^{4}}} = 9.81\,m
\end{equation}
\begin{equation}
	\label{eq:ustar}
	u^\star =  0.00103005\,m \left( z_0 + 6.2 \right) ^{4}
\end{equation}

\section{Linearizing About the Equilibrium Point}

\subsection{Modal Matrix}

Equations $f_1$ and $f_2$ from \ref{eq:syseq1} and \ref{eq:syseq2} may be used to obtain a linearized modal matrix.
The modal matrix general form is given in equation \ref{eq:matAraw}.
Using equations \ref{eq:fderivA}, this gives \ref{eq:matA}.

\begin{equation}
	\label{eq:matAraw}
	{\bf A} = 
	\left[
		\begin{array}{cc}
			 {\frac {df_1}{dx_1}} & {\frac {df_1}{dx_2}} \\
			 {\frac {df_2}{dx_1}} & {\frac {df_2}{dx_2}} \\
		\end{array}
	\right]\bigg|_{x^\star, u^\star}
\end{equation}

\begin{align}
	\label{eq:fderivA}
	 {\frac {df_1}{dx_1}} &= 0 &  {\frac {df_1}{dx_2}} &=1 &  {\frac {df_2}{dx_1}} &= - \,{\frac {38095.23810\,u}{m\left( x_1 + 6.2 \right) ^{5}}} &  {\frac {df_2}{dx_2}} &=0
\end{align}

\begin{equation}
	\label{eq:matA}
	{\bf A} = 
	\left[
		\begin{array}{cc}
			 0 & 1 \\
			 -{\frac {39.24}{z_0 + 6.2}} & 0 \\
		\end{array}
	\right]
\end{equation}


\subsection{Input Matrix}

Similarly, $f_1$ and $f_2$ from \ref{eq:syseq1} and \ref{eq:syseq2} may be used to obtain a linearized input matrix.
The input matrix general form is given in equation \ref{eq:matBraw}.
Using equations \ref{eq:fderivB}, this gives \ref{eq:matB}.

\begin{equation}
	\label{eq:matBraw}
	{\bf B} = 
	\left[
		\begin{array}{c}
			 {\frac {df_1}{du}} \\
			 {\frac {df_2}{du}} \\
		\end{array}
	\right]\bigg|_{x^\star, u^\star}
\end{equation}

\begin{align}
	\label{eq:fderivB}
	 {\frac {df_1}{du}} &= 0 &  {\frac {df_2}{du}} &= {\frac {9523.809524}{m\,\left( x_1 + 6.2 \right) ^{4}}}
\end{align}

\begin{equation}
	\label{eq:matB}
	{\bf B} = 
	\left[
		\begin{array}{c}
			 0 \\
			{\frac {9523.809524}{m\,\left( z_0 + 6.2 \right) ^{4}}} \\
		\end{array}
	\right]
\end{equation}

\subsection{Output Matrix}

The output matrix is simply given by equation \ref{eq:matC}

\begin{equation}
	\label{eq:matC}
	{\bf C} = 
	\left[
		\begin{array}{cc}
			 1 & 0 \\
		\end{array}
	\right]
\end{equation}

\section{Linearized System Model}

The final linearized model of the is given by equations \ref{eq:finalLinearizedx}. %and \ref{eq:finalLinearizedy}.

%TODO: finish

\begin{equation}
	\label{eq:finalLinearizedx}
	{\bf \dot{x}} - {\bf \dot{x}^\star} = 
	\left[
		\begin{array}{cc}
			 0 & 1 \\
			 -{\frac {39.24}{z_0 + 6.2}} & 0 \\
		\end{array}
	\right]
	\left(
		{\bf x} - {\bf x^\star}
	\right)
	+ 
	\left[
		\begin{array}{c}
			 0 \\
			{\frac {9523.809524}{m\,\left( z_0 + 6.2 \right) ^{4}}} \\
		\end{array}
	\right]
	\left(
		u - u^\star
	\right)
\end{equation}