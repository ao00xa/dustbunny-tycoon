\chapter{Nonlinear Model} 

\section{System Dynamic}

From figure \ref{fig:projSys}, summing the forces yields equation \ref{eq:fsum}

\begin{equation}
	\label{eq:fsum}
	m\ddot{z} = F_{mag} - c\dot{z} - mg
\end{equation}

The force $F_{mag}$ is given by \ref{eq:mag} as presented by the manual~\cite{manual}.
The constants $a$, $b$, and $N$ are empirically determined properties of the system. 
For the purposes of this report their values are assumed based on a similar system~\cite{paper} and given in table \ref{tbl:magic}.

\begin{equation}
	\label{eq:mag}
	F_{mag} = {\frac {i}{a \left(z+b \right) ^{N}}}
\end{equation}

\begin{table}[h]
	\centering
	\caption{System properties~\cite{paper}}
	\label{tbl:magic}

	\vspace{6pt}
	\begin{tabular}{|c|c|}
		\hline
		%\vspace{2pt}
		a & 0.000105 \\
		\hline
		%\vspace{2pt}
		b & 6.2 \\
		\hline
		N & 4 \\
		\hline
	\end{tabular}
\end{table}

It is assumed friction and air resistance are negligible, so $c=0$ in equation \ref{eq:fsum}. 
As such, equations \ref{eq:fsum} and \ref{eq:mag} give equation \ref{eq:sysdynam}

\begin{equation}
	\label{eq:sysdynam}
	m\ddot{z} = {\frac {i_{1}}{0.000105\,\left(z+6.2 \right) ^{4}}} - 9.81\,m
\end{equation}

\section{State Space Model}

Using the state variables defined in section \ref{sec:statevars}, two system equations may be written.
These are presented in equations \ref{eq:syseq1} and \ref{eq:syseq2}.

\begin{equation}
	\label{eq:syseq1}
	\dot{x_1} = f_1 = x_2
\end{equation}

\begin{equation}
	\label{eq:syseq2}
	\dot{x_2} = f_2 = {\frac {u}{0.000105\,m\,\left(x_1 + 6.2 \right) ^{4}}} - 9.81
\end{equation}