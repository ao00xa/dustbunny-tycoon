\section{Observer}

An observer was designed to estimate state variable based on the output of the actual system.

\subsection{System Model}

\begin{figure}[h]
    \centering
    \includegraphics[width=0.4\textwidth]{observer}
    \caption{Observer Block Diagram}
    \label{fig:observer}
\end{figure}

The observer block diagram is shown in figure \ref{fig:observer}.
Estimated state variables are denoted by ${\bf \hat{x}}$.
The dynamic of the estimated state variables is shown in \ref{eq:obEstDynamic}.
The error dynamic (with error defined in \ref{eq:obErr}) is shown in equation \ref{eq:obErrDynamic}.

\begin{equation}
	\label{eq:obEstDynamic}
	{\bf \dot{\hat{x}}} = ({\bf A} - {\bf B\,k} - {\bf L\,C})\,{\bf \hat{x}} - {\bf L}\,y
\end{equation}

\begin{equation}
	\label{eq:obErr}
	e = x - \hat{x}
\end{equation}

\begin{equation}
	\label{eq:obErrDynamic}
	{\bf \dot{e}} = ({\bf A} - {\bf L\,C})\,{\bf e}
\end{equation}

\subsection{Gain Determination}

The elements of the {\bf L} matrix were chosen to place the poles of the $({\bf A} - {\bf L\,C})$ matrix.
In order for the observer to track the system state variables, the poles of the error dynamic should be placed in the left hand plane.
In addition, the error dynamic should have a faster response than system with state feedback.
As such, the poles were chosen to be four time larger than the poles of the state feedback system.
This resulted in a stable response for the estimated state variables, $({\bf A} - {\bf B\,k} - {\bf L\,C})$ as show in the figure \ref{fig:obMagResp}, the magnitude response.
This gives elements of the L matrix as $L_1 = 32$ and $L_2 = 249.67$

\begin{figure}[h]
    \centering
    \includegraphics[width=0.4\textwidth]{observerMagnitude}
    \caption{Estimated State Variable Magnitude Response}
    \label{fig:obMagResp}
\end{figure}

